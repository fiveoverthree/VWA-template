\documentclass[12pt, ngerman]{article}
\usepackage{tikz}
\usepackage{graphicx}
\usepackage[authordate, backend=biber]{biblatex-chicago}
% \usepackage[ngerman]{babel}
\usepackage{babel}
\usepackage{eso-pic}
\usepackage[utf8]{inputenc}
\usepackage[a4paper, lmargin=3.5cm, rmargin=2cm]{geometry}
\usepackage{hyperref}
\hypersetup{colorlinks=true, allcolors=black}
%Allow editing of TOF and TOT
\usepackage[titles]{tocloft}
\usepackage[none]{tocbibind}

% make Abbildungsverzeichnis have a section number
\renewcommand{\listoffigures}{\begingroup
	\tocsection
	\tocfile{\listfigurename}{lof}
\endgroup}

% Hinzufügen der Quellen
% Fuer eine einfachere Verwendung ist es empfehlenswert diese Datei mithilfe
% von Zotero & /betterbiblatex auf dem neuesten Stand zu halten
\addbibresource{Quellen.bib}

\begin{document}
% Einfügen des Titelblatts
% Um das Titelblatt zu editieren, müssen in
% titel.tex die entsprechenden Werte geändert werden
% Definition der Zitierregeln. Verwendung mit \izit und \wzit fuer inhaltliche bzw. woertliche Zitate
\newcommand{\izit}[2][]{\ifthenelse{\equal{#1}{}}{(vgl. \cite{#2})}{(vgl. \cite{#2}, #1)}}
\newcommand{\wzit}[2][]{\ifthenelse{\equal{#1}{}}{(\cite{#2})}{(\cite{#2}, #1)}}
% Der Abstract muss wie eine normale Überschrift formatiert sein, darf
% jedoch nicht nummeriert sein
\renewcommand{\abstract}{\section*{Abstract}}

\newgeometry{top=2cm, bottom=2cm}

% To make the logo work
\newcommand\AtPageUpperRight[1]{\AtPageUpperLeft{%
   \makebox[0.9\paperwidth][r]{#1}}}

%overwrite when importing to change Title
\ifdef{\VWAtitel}{}{
    \newcommand{\VWAtitel}{Titel}
}
\ifdef{\VWAauthor}{}{
    \newcommand{\VWAauthor}{Author}
}
\ifdef{\logo}{}{
    \newcommand{\logo}{logo.png}
}
\ifdef{\klasse}{}{
    \newcommand{\klasse}{Klasse}
}
\ifdef{\jahr}{}{
    \newcommand{\jahr}{Jahr}
}
\ifdef{\vorlagedatum}{}{
    \newcommand{\vorlagedatum}{tt.mm.jjjj}
}
\ifdef{\schulinfo}{}{
    \newcommand{\schulinfo}{
    Bundesgymnasium und Bundesrealgymnasium Wien 4 \\
    Wiedner Gymnasium/Sir Karl Popper Schule \\
    A-1040 Wien, Wiedner Gürtel 68
    }
}
\ifdef{\betreuungslehrperson}{}{
    \newcommand{\betreuungslehrperson}{Titel Vorname Nachname}
}

\begin{titlepage}
    \AddToShipoutPictureBG*{%
    \AtPageUpperRight{\raisebox{-2\height}{
        \includegraphics[width=5cm]{\logo}
    }}}

    \begin{center}
        \vspace*{1cm}

        \large{Vorwissenschaftliche Arbeit im Rahmen der Reifeprüfung}

        \vspace*{8cm}
        \huge{\textbf{\VWAtitel}}

        \LARGE{\textbf{\VWAauthor}}

        \large{\klasse\ \jahr}

        \vfill

        \schulinfo

        \vspace*{0.5cm}

        Betreuungslehrperson: \betreuungslehrperson

        \vspace*{0.5cm}
        Vorgelegt am \vorlagedatum

    \end{center}
\end{titlepage}
\restoregeometry


% Nummerierung zuerst mit roemischen Zahlen, danach mit
% arabischen
\pagenumbering{Roman}
\setcounter{page}{2}

\abstract
\clearpage
\tableofcontents
\clearpage
\pagenumbering{arabic}
\section{Einleitung}
\clearpage

\section{Test der Zitate}
Wörtliches Zitat: \wzit[10]{panganos14} \\
Inhaltliches Zitat: \izit[10]{panganos14}

\section{Beispielabbildungen}
\citedfigure[20]{falconerfractal}{Dies ist die Bildunterschrift}{\includegraphics[width=0.8\textwidth]{logo.png}}{test}

\section{Beispieltabelle}
\begin{table}[h!]
    \centering
    \begin{tabular}{|c|c|}
        A & B \\ \hline
        C & D
    \end{tabular}
    \caption{Tolle Tabelle}
\end{table}


% TOC und Bibliography müssen auf eigenen Seiten sein
\clearpage
\printbibliography[title=\section{Literaturverzeichnis}]
\listoffigures
\listoftables
\end{document}

