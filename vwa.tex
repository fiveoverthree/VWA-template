\documentclass[12pt]{article}
\usepackage{tikz}
\usepackage{graphicx}
% \usepackage{fancyhdr}
\usepackage[authordate, backend=biber]{biblatex-chicago}
\usepackage{ngerman}
\usepackage{eso-pic}
\usepackage[utf8]{inputenc}
\usepackage[a4paper, lmargin=3.5cm, rmargin=2cm]{geometry}
\usepackage{hyperref}
\hypersetup{colorlinks=true, allcolors=black}

% Definition der Zitierregeln. Verwendung mit \izit und \wzit fuer inhaltliche bzw. woertliche Zitate
\newcommand{\izit}[2]{(vgl. \cite{#1}, #2)}
\newcommand{\wzit}[2]{(\cite{#1}, #2)}
% Der Abstract muss wie eine normale Überschrift formatiert sein, darf
% jedoch nicht nummeriert sein
\renewcommand{\abstract}{\section*{Abstract}}

% Hinzufügen der Quellen
% Fuer eine einfachere Verwendung ist es empfehlenswert diese Datei mithilfe
% von Zotero & /betterbiblatex auf dem neuesten Stand zu halten
\addbibresource{Quellen.bib}

\begin{document}
% Einfügen des Titelblatts
% Um das Titelblatt zu editieren, müssen in
% titel.tex die entsprechenden Werte geändert werden

\newgeometry{top=2cm, bottom=2cm}

% To make the logo work
\newcommand\AtPageUpperRight[1]{\AtPageUpperLeft{%
   \makebox[0.9\paperwidth][r]{#1}}}

\begin{titlepage}
    \AddToShipoutPictureBG*{%
    \AtPageUpperRight{\raisebox{-2\height}{
        \includegraphics[width=5cm]{logo.png}
    }}}

    \begin{center}
        \vspace*{1cm}

        \large{Vorwissenschaftliche Arbeit im Rahmen der Reifeprüfung}

        \vspace*{8cm}
        \huge{\textbf{Titel}}

        \LARGE{\textbf{Author}}

        \large{Klasse Jahr}

        \vfill

        Bundesgymnasium und Bundesrealgymnasium Wien 4 \\
        Wiedner Gymnasium/Sir Karl Popper Schule \\
        A-1040 Wien, Wiedner Gürtel 68

        \vspace*{0.5cm}

        Betreuungslehrperson: Titel Vorname Nachname

        \vspace*{0.5cm}
        Vorgelegt am tt.mm.jjjj

    \end{center}
\end{titlepage}
\restoregeometry


\abstract
\clearpage
\tableofcontents
\clearpage

\section{Test der Zitate}
Wörtliches Zitat: \wzit{panganos14}{5} \\
Inhaltliches Zitat: \izit{panganos14}{10}

\section{Beispielabbildungen}
\begin{figure}[h!]
    \includegraphics[width=0.8\textwidth]{logo.png}
    \caption{Tolles Logo des Wiedner Gymnasiums / der Sir Karl Popper Schule \wzit{author_fig}{}}
\end{figure}

\section{Beispieltabelle}
\begin{table}[h!]
    \centering
    \begin{tabular}{|c|c|}
        A & B \\ \hline
        C & D
    \end{tabular}
    \caption{Tolle Tabelle}
\end{table}


% TOC und Bibliography müssen auf eigenen Seiten sein
\clearpage
\printbibliography[title=\section{Literaturverzeichnis}]
\listoffigures
\listoftables
\end{document}

